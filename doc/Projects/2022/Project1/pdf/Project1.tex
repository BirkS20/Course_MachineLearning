%%
%% Automatically generated file from DocOnce source
%% (https://github.com/doconce/doconce/)
%% doconce format latex Project1.do.txt --print_latex_style=trac --latex_admon=paragraph
%%


%-------------------- begin preamble ----------------------

\documentclass[%
oneside,                 % oneside: electronic viewing, twoside: printing
final,                   % draft: marks overfull hboxes, figures with paths
10pt]{article}

\listfiles               %  print all files needed to compile this document

\usepackage{relsize,makeidx,color,setspace,amsmath,amsfonts,amssymb}
\usepackage[table]{xcolor}
\usepackage{bm,ltablex,microtype}

\usepackage[pdftex]{graphicx}

\usepackage[T1]{fontenc}
%\usepackage[latin1]{inputenc}
\usepackage{ucs}
\usepackage[utf8x]{inputenc}

\usepackage{lmodern}         % Latin Modern fonts derived from Computer Modern

% Hyperlinks in PDF:
\definecolor{linkcolor}{rgb}{0,0,0.4}
\usepackage{hyperref}
\hypersetup{
    breaklinks=true,
    colorlinks=true,
    linkcolor=linkcolor,
    urlcolor=linkcolor,
    citecolor=black,
    filecolor=black,
    %filecolor=blue,
    pdfmenubar=true,
    pdftoolbar=true,
    bookmarksdepth=3   % Uncomment (and tweak) for PDF bookmarks with more levels than the TOC
    }
%\hyperbaseurl{}   % hyperlinks are relative to this root

\setcounter{tocdepth}{2}  % levels in table of contents

% prevent orhpans and widows
\clubpenalty = 10000
\widowpenalty = 10000

% --- end of standard preamble for documents ---


% insert custom LaTeX commands...

\raggedbottom
\makeindex
\usepackage[totoc]{idxlayout}   % for index in the toc
\usepackage[nottoc]{tocbibind}  % for references/bibliography in the toc

%-------------------- end preamble ----------------------

\begin{document}

% matching end for #ifdef PREAMBLE

\newcommand{\exercisesection}[1]{\subsection*{#1}}


% ------------------- main content ----------------------

\subsection*{Format for electronic delivery of report and programs}

The preferred format for the report is a PDF file. You can also use DOC or postscript formats or as an ipython notebook file.  As programming language we prefer that you choose between C/C++, Fortran2008, Julia or Python. The following prescription should be followed when preparing the report:

\begin{itemize}
  \item Use Canvas to hand in your projects, log in  at  \href{{https://www.uio.no/english/services/it/education/canvas/}}{\nolinkurl{https://www.uio.no/english/services/it/education/canvas/}} with your normal UiO username and password.

  \item Upload \textbf{only} the report file or the link to your GitHub/GitLab or similar typo of  repos!  For the source code file(s) you have developed please provide us with your link to your GitHub/GitLab or similar  domain.  The report file should include all of your discussions and a list of the codes you have developed.  Do not include library files which are available at the course homepage, unless you have made specific changes to them.

  \item In your GitHub/GitLab or similar repository, please include a folder which contains selected results. These can be in the form of output from your code for a selected set of runs and input parameters.
\end{itemize}

\noindent
Finally, 
we encourage you to collaborate. Optimal working groups consist of 
2-3 students. You can then hand in a common report. 

\subsection*{Software and needed installations}

If you have Python installed (we recommend Python3) and you feel pretty familiar with installing different packages, 
we recommend that you install the following Python packages via \textbf{pip} as
\begin{enumerate}
\item pip install numpy scipy matplotlib ipython scikit-learn tensorflow sympy pandas pillow
\end{enumerate}

\noindent
For Python3, replace \textbf{pip} with \textbf{pip3}.

See below for a discussion of \textbf{tensorflow} and \textbf{scikit-learn}. 

For OSX users we recommend also, after having installed Xcode, to install \textbf{brew}. Brew allows 
for a seamless installation of additional software via for example
\begin{enumerate}
\item brew install python3
\end{enumerate}

\noindent
For Linux users, with its variety of distributions like for example the widely popular Ubuntu distribution
you can use \textbf{pip} as well and simply install Python as 
\begin{enumerate}
\item sudo apt-get install python3  (or python for python2.7)
\end{enumerate}

\noindent
etc etc. 

If you don't want to install various Python packages with their dependencies separately, we recommend two widely used distrubutions which set up  all relevant dependencies for Python, namely
\begin{enumerate}
\item \href{{https://docs.anaconda.com/}}{Anaconda} Anaconda is an open source distribution of the Python and R programming languages for large-scale data processing, predictive analytics, and scientific computing, that aims to simplify package management and deployment. Package versions are managed by the package management system \textbf{conda}

\item \href{{https://www.enthought.com/product/canopy/}}{Enthought canopy}  is a Python distribution for scientific and analytic computing distribution and analysis environment, available for free and under a commercial license.
\end{enumerate}

\noindent
Popular software packages written in Python for ML are

\begin{itemize}
\item \href{{http://scikit-learn.org/stable/}}{Scikit-learn}, 

\item \href{{https://www.tensorflow.org/}}{Tensorflow},

\item \href{{http://pytorch.org/}}{PyTorch} and 

\item \href{{https://keras.io/}}{Keras}.
\end{itemize}

\noindent
These are all freely available at their respective GitHub sites. They 
encompass communities of developers in the thousands or more. And the number
of code developers and contributors keeps increasing.


% ------------------- end of main content ---------------

\end{document}

