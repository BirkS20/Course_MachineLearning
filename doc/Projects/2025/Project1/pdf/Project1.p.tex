%%
%% Automatically generated file from DocOnce source
%% (https://github.com/doconce/doconce/)
%% doconce format latex Project1.do.txt --print_latex_style=trac --latex_admon=paragraph
%%
% #ifdef PTEX2TEX_EXPLANATION
%%
%% The file follows the ptex2tex extended LaTeX format, see
%% ptex2tex: https://code.google.com/p/ptex2tex/
%%
%% Run
%%      ptex2tex myfile
%% or
%%      doconce ptex2tex myfile
%%
%% to turn myfile.p.tex into an ordinary LaTeX file myfile.tex.
%% (The ptex2tex program: https://code.google.com/p/ptex2tex)
%% Many preprocess options can be added to ptex2tex or doconce ptex2tex
%%
%%      ptex2tex -DMINTED myfile
%%      doconce ptex2tex myfile envir=minted
%%
%% ptex2tex will typeset code environments according to a global or local
%% .ptex2tex.cfg configure file. doconce ptex2tex will typeset code
%% according to options on the command line (just type doconce ptex2tex to
%% see examples). If doconce ptex2tex has envir=minted, it enables the
%% minted style without needing -DMINTED.
% #endif

% #define PREAMBLE

% #ifdef PREAMBLE
%-------------------- begin preamble ----------------------

\documentclass[%
oneside,                 % oneside: electronic viewing, twoside: printing
final,                   % draft: marks overfull hboxes, figures with paths
10pt]{article}

\listfiles               %  print all files needed to compile this document

\usepackage{relsize,makeidx,color,setspace,amsmath,amsfonts,amssymb}
\usepackage[table]{xcolor}
\usepackage{bm,ltablex,microtype}

\usepackage[pdftex]{graphicx}

\usepackage[T1]{fontenc}
%\usepackage[latin1]{inputenc}
\usepackage{ucs}
\usepackage[utf8x]{inputenc}

\usepackage{lmodern}         % Latin Modern fonts derived from Computer Modern

% Hyperlinks in PDF:
\definecolor{linkcolor}{rgb}{0,0,0.4}
\usepackage{hyperref}
\hypersetup{
    breaklinks=true,
    colorlinks=true,
    linkcolor=linkcolor,
    urlcolor=linkcolor,
    citecolor=black,
    filecolor=black,
    %filecolor=blue,
    pdfmenubar=true,
    pdftoolbar=true,
    bookmarksdepth=3   % Uncomment (and tweak) for PDF bookmarks with more levels than the TOC
    }
%\hyperbaseurl{}   % hyperlinks are relative to this root

\setcounter{tocdepth}{2}  % levels in table of contents

% prevent orhpans and widows
\clubpenalty = 10000
\widowpenalty = 10000

% --- end of standard preamble for documents ---


% insert custom LaTeX commands...

\raggedbottom
\makeindex
\usepackage[totoc]{idxlayout}   % for index in the toc
\usepackage[nottoc]{tocbibind}  % for references/bibliography in the toc

%-------------------- end preamble ----------------------

\begin{document}

% matching end for #ifdef PREAMBLE
% #endif

\newcommand{\exercisesection}[1]{\subsection*{#1}}


% ------------------- main content ----------------------



% ----------------- title -------------------------

\thispagestyle{empty}

\begin{center}
{\LARGE\bf
\begin{spacing}{1.25}
Project 1 on Machine Learning, deadline October 6 (midnight), 2025
\end{spacing}
}
\end{center}

% ----------------- author(s) -------------------------

\begin{center}
{\bf Data Analysis and Machine Learning FYS-STK3155/FYS4155}
\end{center}

    \begin{center}
% List of all institutions:
\centerline{{\small University of Oslo, Norway}}
\end{center}
    
% ----------------- end author(s) -------------------------

% --- begin date ---
\begin{center}
September 2
\end{center}
% --- end date ---

\vspace{1cm}


\subsection{Preamble: Note on writing reports, using reference material, AI and other tools}

We want you to answer the three different projects by handing in
reports written like a standard scientific/technical report.  The links
at \href{{https://github.com/CompPhysics/MachineLearning/tree/master/doc/Projects}}{\nolinkurl{https://github.com/CompPhysics/MachineLearning/tree/master/doc/Projects}}
Furthermore, at the same link, 
you can find examples of previous reports. How to write reports will
also be discussed during the various lab sessions. Please do ask us if you are in doubt.

When using codes and material from other sources, you should refer to
these in the bibliography of your report, indicating wherefrom you for
example got the code, whether this is from the lecture notes,
softwares like Scikit-Learn, TensorFlow, PyTorch or other sources such
AI software. These should always be cited correctly. How to cite some
of the libraries is often indicated from their corresponding GitHub
sites or websites, see for example how to cite Scikit-Learn at
\href{{https://scikit-learn.org/dev/about.html}}{\nolinkurl{https://scikit-learn.org/dev/about.html}}.

We enocurage you to use tools like
\href{{https://openai.com/chatgpt/}}{ChatGPT} or similar in writing the report. If you use for example ChatGPT,
please do cite it properly and include (if possible) your questions and answers as an addition to the report. This can
be uploaded to for example your website, GitHub/GitLab or similar as supplemental material.

If you would like to study other data sets, feel free to propose other
sets. What we have proposed here are mere suggestions from our
side. If you opt for another data set, consider using a set which has
been studied in the scientific literature. This makes it easier for
you to compare and analyze your results. Comparing with existing
results from the scientific literature is also an essential element of
the scientific discussion.  The University of California at Irvine
with its Machine Learning repository at
\href{{https://archive.ics.uci.edu/ml/index.php}}{\nolinkurl{https://archive.ics.uci.edu/ml/index.php}} is an excellent site to
look up for examples and
inspiration. \href{{https://www.kaggle.com/}}{Kaggle.com} is an equally
interesting site. Feel free to explore these sites. When selecting
other data sets, make sure these are sets used for regression problems
(not classification).

% !split
\subsection{Regression analysis and resampling methods}

The main aim of this project is to study in more detail various
regression methods, including the Ordinary Least Squares (OLS) method.
In addition to the scientific part, in this course we want also to
give you an experience in writing scientific reports.

We will study how to fit polynomials to specific
one-dimensional functions (feel free to replace the suggested function with more complicated ones).

We will use Runge's function (see \href{{https://en.wikipedia.org/wiki/Runge%27s_phenomenon}}{\nolinkurl{https://en.wikipedia.org/wiki/Runge\%27s_phenomenon}} for a discussion).  The one-dimensional function we will study is
\[
f(x) = \frac{1}{1+25x^2}.
\]

Our first step will be to perform an OLS regression analysis of this
function, trying out a polynomial fit with an $x$ dependence of the
form $[x,x^2,\dots]$.  We can use a uniform distribution to set up the
arrays of values for $x \in [-1,1]$, or alternatively use a fixed step size.
Thereafter we will repeat much of the
same procedure using the Ridge and Lasso regression methods,
introducing thus a dependence on the hyperparameter  (penalty) $\lambda$.

We will also include bootstrap as a resampling technique in order to
study the so-called \textbf{bias-variance tradeoff}.  After that we will
include the cross-validation technique.

\paragraph{Part a : Ordinary Least Square (OLS) for the Runge function.}
We will generate our own dataset for a function
$\mathrm{Runge}(x)$ function with $x\in [-1,1]$. You should explore also the addition
of an added stochastic noise to this function using the normal
distribution $N(0,1)$.

\emph{Write your own code} (using for example the  pseudoinverse function \textbf{pinv} from  \textbf{Numpy} ) and perform a standard \textbf{ordinary least square regression}
analysis using polynomials in $x$ up to  order $15$. Explore the dependence on the number of data points and the polynomial degree.

Evaluate the mean Squared error (MSE)

\[ MSE(\bm{y},\tilde{\bm{y}}) = \frac{1}{n}
\sum_{i=0}^{n-1}(y_i-\tilde{y}_i)^2, 
\] 

and the $R^2$ score function.  If $\tilde{\bm{y}}_i$ is the predicted
value of the $i-th$ sample and $y_i$ is the corresponding true value,
then the score $R^2$ is defined as

\[
R^2(\bm{y}, \tilde{\bm{y}}) = 1 - \frac{\sum_{i=0}^{n - 1} (y_i - \tilde{y}_i)^2}{\sum_{i=0}^{n - 1} (y_i - \bar{y})^2},
\]

where we have defined the mean value  of $\bm{y}$ as

\[
\bar{y} =  \frac{1}{n} \sum_{i=0}^{n - 1} y_i.
\]

Plot the resulting scores (MSE and R$^2$) as functions of the polynomial degree (here up to polymial degree 20).
Plot also the parameters $\theta$ as you increase the order of the polynomial. Comment your results.

Your code has to include a scaling/centering of the data (for example by
subtracting the mean value), and
a split of the data in training and test data. For this exercise you can
either write your own code or use for example the function for
splitting training data provided by the library \textbf{Scikit-Learn} (make
sure you have installed it).  This function is called
$train\_test\_split$.  \textbf{You should present a critical discussion of why and how you have scaled or not scaled the data}.

It is normal in essentially all Machine Learning studies to split the
data in a training set and a test set (eventually  also an additional
validation set).  There
is no explicit recipe for how much data should be included as training
data and say test data.  An accepted rule of thumb is to use
approximately $2/3$ to $4/5$ of the data as training data.

You can easily reuse the solutions to your exercises from week 35.
See also the lecture slides from week 35 and week 36.

On scaling, we recommend reading the following section from the scikit-learn software description, see \href{{https://scikit-learn.org/stable/auto_examples/preprocessing/plot_all_scaling.html#plot-all-scaling-standard-scaler-section}}{\nolinkurl{https://scikit-learn.org/stable/auto_examples/preprocessing/plot_all_scaling.html\#plot-all-scaling-standard-scaler-section}}.

\paragraph{Part b: Adding Ridge regression for  the Runge  function.}
Write your own code for the Ridge method as done in the previous
exercise. The lecture notes from week 35 and 36 contain more information. Furthermore, the  exercise from week 36 is something you can reuse here.

Perform the same analysis as you did in the previous exercise but now for different values of $\lambda$. Compare and
analyze your results with those obtained in part a) with the ordinary least squares method. Study the
dependence on $\lambda$.

\paragraph{Part c: Writing your own gradient descent code.}
Replace now the analytical expressions for the optimal parameters
$\bm{\theta}$ with your own gradient descent code. In this exercise we
focus only on the simplest gradient descent approach with a fixed
learning rate (see the exercises from week 37 and the lecture notes
from week 36).

Study and compare your results from parts a) and b) with your gradient
descent approch. Discuss in particular the role of the learning rate.

\paragraph{Part d: Including momentum and more advanced ways to update the learning the rate.}
We keep our focus on OLS and Ridge regression and update our code for
the gradient descent method by including \textbf{momentum}, \textbf{ADAgrad},
\textbf{RMSprop} and \textbf{ADAM} as methods fro iteratively updating your learning
rate. Discuss the results and compare the different methods applied to
the one-dimensional Runge function.

\paragraph{Part e: Writing our own code for Lasso regression.}
LASSO regression (see lecture slides from week 36 and week 37)
represents our first encounter with a machine learning method which
cannot be solved through analytical expressions. Use the gradient
descent methods you developed in parts c) and d) to solve the LASSO
optimization problem. You can compare your results using
the functionalities of \textbf{Scikit-Learn}.

Discuss (critically) your results for the Runge function from OLS,
Ridge and LASSO regression using the various gradient descent
approaches.

\paragraph{Part f: Stochastic gradient descent.}
Our last  gradient step is to include stochastic gradient descent using the
same methods to update the learning rates as in parts c-e).
Compare and discuss your results with and without stochastic gradient and give a critical assessment of the various methods.

\paragraph{Part g: Bias-variance trade-off and resampling techniques.}
Our aim here is to study the bias-variance trade-off by implementing the \textbf{bootstrap} resampling technique.
\textbf{We will only use the simpler ordinary least squares here}.

With a code which does OLS and includes resampling techniques, 
we will now discuss the bias-variance trade-off in the context of
continuous predictions such as regression. However, many of the
intuitions and ideas discussed here also carry over to classification
tasks and basically all Machine Learning algorithms. 

Before you perform an analysis of the bias-variance trade-off on your test data, make
first a figure similar to Fig.~2.11 of Hastie, Tibshirani, and
Friedman. Figure 2.11 of this reference displays only the test and training MSEs. The test MSE can be used to 
indicate possible regions of low/high bias and variance. You will most likely not get an
equally smooth curve!

With this result we move on to the bias-variance trade-off analysis.

Consider a
dataset $\mathcal{L}$ consisting of the data
$\mathbf{X}_\mathcal{L}=\{(y_j, \boldsymbol{x}_j), j=0\ldots n-1\}$.

As in part d), we assume that the true data is generated from a noisy model

\[
\bm{y}=f(\boldsymbol{x}) + \bm{\epsilon}.
\]

Here $\epsilon$ is normally distributed with mean zero and standard
deviation $\sigma^2$.

In our derivation of the ordinary least squares method we defined then
an approximation to the function $f$ in terms of the parameters
$\bm{\beta}$ and the design matrix $\bm{X}$ which embody our model,
that is $\bm{\tilde{y}}=\bm{X}\bm{\beta}$.

The parameters $\bm{\beta}$ are in turn found by optimizing the mean
squared error via the so-called cost function

\[
C(\bm{X},\bm{\beta}) =\frac{1}{n}\sum_{i=0}^{n-1}(y_i-\tilde{y}_i)^2=\mathbb{E}\left[(\bm{y}-\bm{\tilde{y}})^2\right].
\]
Here the expected value $\mathbb{E}$ is the sample value. 

Show that you can rewrite  this in terms of a term which contains the variance of the model itself (the so-called variance term), a
term which measures the deviation from the true data and the mean value of the model (the bias term) and finally the variance of the noise.
That is, show that
\[
\mathbb{E}\left[(\bm{y}-\bm{\tilde{y}})^2\right]=\mathrm{Bias}[\tilde{y}]+\mathrm{var}[\tilde{y}]+\sigma^2, 
\]
with 
\[
\mathrm{Bias}[\tilde{y}]=\mathbb{E}\left[\left(\bm{y}-\mathbb{E}\left[\bm{\tilde{y}}\right]\right)^2\right],
\]
and 
\[
\mathrm{var}[\tilde{y}]=\mathbb{E}\left[\left(\tilde{\bm{y}}-\mathbb{E}\left[\bm{\tilde{y}}\right]\right)^2\right]=\frac{1}{n}\sum_i(\tilde{y}_i-\mathbb{E}\left[\bm{\tilde{y}}\right])^2.
\]
The answer to this exercise should be included in the theory part of the report.  This exercise is also part of the weekly exercises of week 38.
Explain what the terms mean and discuss their interpretations.

Perform then a bias-variance analysis of the Runge function by
studying the MSE value as function of the complexity of your model.

Discuss the bias and variance trade-off as function
of your model complexity (the degree of the polynomial) and the number
of data points, and possibly also your training and test data using the \textbf{bootstrap} resampling method.
You can follow the code example in the jupyter-book at \href{{https://compphysics.github.io/MachineLearning/doc/LectureNotes/_build/html/chapter3.html#the-bias-variance-tradeoff}}{\nolinkurl{https://compphysics.github.io/MachineLearning/doc/LectureNotes/_build/html/chapter3.html\#the-bias-variance-tradeoff}}.

\paragraph{Part h):  Cross-validation as resampling techniques, adding more complexity.}
The aim here is to implement another widely popular
resampling technique, the so-called cross-validation method.  

Implement the $k$-fold cross-validation algorithm (feel free to use the functionality of \textbf{Scikit-Learn} or write your own code) and evaluate again the MSE function resulting
from the test folds. 

Compare the MSE you get from your cross-validation code with the one
you got from your \textbf{bootstrap} code. Comment your results. Try $5-10$
folds.  

In addition to using the ordinary least squares method, you should include both Ridge and Lasso regression in the analysis. 

\subsection{Background literature}

\begin{enumerate}
\item For a discussion and derivation of the variances and mean squared errors using linear regression, see the \href{{https://arxiv.org/abs/1509.09169}}{Lecture notes on ridge regression by Wessel N. van Wieringen}

\item The textbook of \href{{https://www.springer.com/gp/book/9780387848570}}{Trevor Hastie, Robert Tibshirani, Jerome H. Friedman, The Elements of Statistical Learning, Springer}, chapters 3 and 7 are the most relevant ones for the analysis here. 
\end{enumerate}

\noindent
\subsection{Introduction to numerical projects}

Here follows a brief recipe and recommendation on how to answer the various questions when preparing your answers. 

\begin{itemize}
  \item Give a short description of the nature of the problem and the eventual  numerical methods you have used.

  \item Describe the algorithm you have used and/or developed. Here you may find it convenient to use pseudocoding. In many cases you can describe the algorithm in the program itself.

  \item Include the source code of your program. Comment your program properly. You should have the code at your GitHub/GitLab link. You can also place the code in an appendix of your report.

  \item If possible, try to find analytic solutions, or known limits in order to test your program when developing the code.

  \item Include your results either in figure form or in a table. Remember to        label your results. All tables and figures should have relevant captions        and labels on the axes.

  \item Try to evaluate the reliabilty and numerical stability/precision of your results. If possible, include a qualitative and/or quantitative discussion of the numerical stability, eventual loss of precision etc.

  \item Try to give an interpretation of you results in your answers to  the problems.

  \item Critique: if possible include your comments and reflections about the  exercise, whether you felt you learnt something, ideas for improvements and  other thoughts you've made when solving the exercise. We wish to keep this course at the interactive level and your comments can help us improve it.

  \item Try to establish a practice where you log your work at the  computerlab. You may find such a logbook very handy at later stages in your work, especially when you don't properly remember  what a previous test version  of your program did. Here you could also record  the time spent on solving the exercise, various algorithms you may have tested or other topics which you feel worthy of mentioning.
\end{itemize}

\noindent
\subsection{Format for electronic delivery of report and programs}

The preferred format for the report is a PDF file. You can also use DOC or postscript formats or as an ipython notebook file.  As programming language we prefer that you choose between C/C++, Fortran2008, Julia or Python. The following prescription should be followed when preparing the report:

\begin{itemize}
  \item Use Canvas to hand in your projects, log in  at  \href{{https://www.uio.no/english/services/it/education/canvas/}}{\nolinkurl{https://www.uio.no/english/services/it/education/canvas/}} with your normal UiO username and password.

  \item Upload \textbf{only} the report file or the link to your GitHub/GitLab or similar typo of  repos!  For the source code file(s) you have developed please provide us with your link to your GitHub/GitLab or similar  domain.  The report file should include all of your discussions and a list of the codes you have developed.  Do not include library files which are available at the course homepage, unless you have made specific changes to them.

  \item In your GitHub/GitLab or similar repository, please include a folder which contains selected results. These can be in the form of output from your code for a selected set of runs and input parameters.
\end{itemize}

\noindent
Finally, 
we encourage you to collaborate. Optimal working groups consist of 
2-3 students. You can then hand in a common report. 

\subsection{Software and needed installations}

If you have Python installed (we recommend Python3) and you feel pretty familiar with installing different packages, 
we recommend that you install the following Python packages via \textbf{pip} as
\begin{enumerate}
\item pip install numpy scipy matplotlib ipython scikit-learn tensorflow sympy pandas pillow
\end{enumerate}

\noindent
For Python3, replace \textbf{pip} with \textbf{pip3}.

See below for a discussion of \textbf{tensorflow} and \textbf{scikit-learn}. 

For OSX users we recommend also, after having installed Xcode, to install \textbf{brew}. Brew allows 
for a seamless installation of additional software via for example
\begin{enumerate}
\item brew install python3
\end{enumerate}

\noindent
For Linux users, with its variety of distributions like for example the widely popular Ubuntu distribution
you can use \textbf{pip} as well and simply install Python as 
\begin{enumerate}
\item sudo apt-get install python3  (or python for python2.7)
\end{enumerate}

\noindent
etc etc. 

If you don't want to install various Python packages with their dependencies separately, we recommend two widely used distrubutions which set up  all relevant dependencies for Python, namely
\begin{enumerate}
\item \href{{https://docs.anaconda.com/}}{Anaconda} Anaconda is an open source distribution of the Python and R programming languages for large-scale data processing, predictive analytics, and scientific computing, that aims to simplify package management and deployment. Package versions are managed by the package management system \textbf{conda}

\item \href{{https://www.enthought.com/product/canopy/}}{Enthought canopy}  is a Python distribution for scientific and analytic computing distribution and analysis environment, available for free and under a commercial license.
\end{enumerate}

\noindent
Popular software packages written in Python for ML are

\begin{itemize}
\item \href{{http://scikit-learn.org/stable/}}{Scikit-learn}, 

\item \href{{https://www.tensorflow.org/}}{Tensorflow},

\item \href{{http://pytorch.org/}}{PyTorch} and 

\item \href{{https://keras.io/}}{Keras}.
\end{itemize}

\noindent
These are all freely available at their respective GitHub sites. They 
encompass communities of developers in the thousands or more. And the number
of code developers and contributors keeps increasing.


% ------------------- end of main content ---------------

% #ifdef PREAMBLE
\end{document}
% #endif

