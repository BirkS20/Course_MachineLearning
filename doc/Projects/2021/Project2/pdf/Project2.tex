%%
%% Automatically generated file from DocOnce source
%% (https://github.com/hplgit/doconce/)
%%
%%


%-------------------- begin preamble ----------------------

\documentclass[%
oneside,                 % oneside: electronic viewing, twoside: printing
final,                   % draft: marks overfull hboxes, figures with paths
10pt]{article}

\listfiles               %  print all files needed to compile this document

\usepackage{relsize,makeidx,color,setspace,amsmath,amsfonts,amssymb}
\usepackage[table]{xcolor}
\usepackage{bm,ltablex,microtype}

\usepackage[pdftex]{graphicx}

\usepackage[T1]{fontenc}
%\usepackage[latin1]{inputenc}
\usepackage{ucs}
\usepackage[utf8x]{inputenc}

\usepackage{lmodern}         % Latin Modern fonts derived from Computer Modern

% Hyperlinks in PDF:
\definecolor{linkcolor}{rgb}{0,0,0.4}
\usepackage{hyperref}
\hypersetup{
    breaklinks=true,
    colorlinks=true,
    linkcolor=linkcolor,
    urlcolor=linkcolor,
    citecolor=black,
    filecolor=black,
    %filecolor=blue,
    pdfmenubar=true,
    pdftoolbar=true,
    bookmarksdepth=3   % Uncomment (and tweak) for PDF bookmarks with more levels than the TOC
    }
%\hyperbaseurl{}   % hyperlinks are relative to this root

\setcounter{tocdepth}{2}  % levels in table of contents

% --- fancyhdr package for fancy headers ---
\usepackage{fancyhdr}
\fancyhf{} % sets both header and footer to nothing
\renewcommand{\headrulewidth}{0pt}
\fancyfoot[LE,RO]{\thepage}
% Ensure copyright on titlepage (article style) and chapter pages (book style)
\fancypagestyle{plain}{
  \fancyhf{}
  \fancyfoot[C]{{\footnotesize \copyright\ 1999-2021, "Data Analysis and Machine Learning FYS-STK3155/FYS4155":"http://www.uio.no/studier/emner/matnat/fys/FYS3155/index-eng.html". Released under CC Attribution-NonCommercial 4.0 license}}
%  \renewcommand{\footrulewidth}{0mm}
  \renewcommand{\headrulewidth}{0mm}
}
% Ensure copyright on titlepages with \thispagestyle{empty}
\fancypagestyle{empty}{
  \fancyhf{}
  \fancyfoot[C]{{\footnotesize \copyright\ 1999-2021, "Data Analysis and Machine Learning FYS-STK3155/FYS4155":"http://www.uio.no/studier/emner/matnat/fys/FYS3155/index-eng.html". Released under CC Attribution-NonCommercial 4.0 license}}
  \renewcommand{\footrulewidth}{0mm}
  \renewcommand{\headrulewidth}{0mm}
}

\pagestyle{fancy}


% prevent orhpans and widows
\clubpenalty = 10000
\widowpenalty = 10000

% --- end of standard preamble for documents ---


% insert custom LaTeX commands...

\raggedbottom
\makeindex
\usepackage[totoc]{idxlayout}   % for index in the toc
\usepackage[nottoc]{tocbibind}  % for references/bibliography in the toc

%-------------------- end preamble ----------------------

\begin{document}

% matching end for #ifdef PREAMBLE

\newcommand{\exercisesection}[1]{\subsection*{#1}}


% ------------------- main content ----------------------



% ----------------- title -------------------------

\thispagestyle{empty}

\begin{center}
{\LARGE\bf
\begin{spacing}{1.25}
Project 2 on Machine Learning, deadline November 15 (Midnight)
\end{spacing}
}
\end{center}

% ----------------- author(s) -------------------------

\begin{center}
{\bf \href{{http://www.uio.no/studier/emner/matnat/fys/FYS3155/index-eng.html}}{Data Analysis and Machine Learning FYS-STK3155/FYS4155}}
\end{center}

    \begin{center}
% List of all institutions:
\centerline{{\small Department of Physics, University of Oslo, Norway}}
\end{center}
    
% ----------------- end author(s) -------------------------

% --- begin date ---
\begin{center}
Oct 12, 2021
\end{center}
% --- end date ---

\vspace{1cm}


\subsection*{Classification and Regression, from linear and logistic regression to neural networks}

The main aim of this project is to study both classification and
regression problems by developing our own feed-forward neural network (FFNN) code. We can reuse the regression algorithms studied
in project 1. We will also include logistic regression for classification
problems and write our own FFNN code for studying
both regression and classification problems.  The codes developed in
project 1, including bootstrap \textbf{and/or} cross-validation as well as the
computation of the mean-squared error and/or the $R2$ or the accuracy score (classification problems) functions can
also be utilized in the present analysis. 

\textbf{Important Note}: This project as well as project 3 have to be written as a scientific report. The instructions on how to do this and how we grade are available at \href{{https://github.com/CompPhysics/MachineLearning/blob/master/doc/Projects/EvaluationGrading/EvaluationForm.md}}{\nolinkurl{https://github.com/CompPhysics/MachineLearning/blob/master/doc/Projects/EvaluationGrading/EvaluationForm.md}}. We will discuss this format during the various lab sessions. Please do spend some time to read our guidelines.

The data sets that we propose here are (the default sets)

\begin{itemize}
\item Regression (fitting a continuous function). In this part you will need to bring back your results from project 1 and compare these with what you get from your Neural Network code to be developed here. The data sets could be
\begin{enumerate}

 \item Either the Franke function or the terrain data from project 1, or data sets your propose.

\end{enumerate}

\noindent
\item Classification. Here you will also need to develop a Logistic regression code that you will use to compare with the Neural Network code. The data set we propose are the so-called \href{{https://www.kaggle.com/uciml/breast-cancer-wisconsin-data}}{Wisconsin Breat Cancer Data} data set of images representing various features of tumors. These are discussed intensively in the lecture notes, see for example the slides from \href{{https://compphysics.github.io/MachineLearning/doc/pub/week41/html/week40.html}}{week 40}. A longer explanation with links to the scientific literature can be found at the \href{{https://archive.ics.uci.edu/ml/datasets/Breast+Cancer+Wisconsin+%28Diagnostic%29}}{Machine Learning repository of the University of California at Irvine}. Feel free to consult this site and the pertinent  literature.
\end{itemize}

\noindent
You can find more information about this at the \href{{https://scikit-learn.org/stable/modules/generated/sklearn.datasets.load_breast_cancer.html}}{Scikit-Learn site} or at the \href{{https://archive.ics.uci.edu/ml/datasets/breast+cancer+wisconsin+(original)}}{University of California at Irvine}. 


However, if you would like to study other data sets, feel free to
propose other sets. What we list here are mere suggestions from our
side. If you opt for another data set, consider using a set which
has been studied in the scientific literature. This makes it easier
for you to compare and analyze your results. Comparing with existing results from the scientific literature  is also an essential
element of the scientific discussion.  The University of California at Irvine with its Machine Learning repository at \href{{https://archive.ics.uci.edu/ml/index.php}}{\nolinkurl{https://archive.ics.uci.edu/ml/index.php}} is an excellent site to look up for examples and inspiration. \href{{https://www.kaggle.com/}}{Kaggle.com} is an equally interesting site. Feel free to explore these sites.


We will start with a regression problem and we will reuse our codes from project 1 starting with writing our own Stochastic Gradient Descent (SGD) code. 

\paragraph{Part a): Write your own Stochastic Gradient Descent  code, first step.}
In order to get started, we will now replace in our standard ordinary
least squares (OLS) and Ridge regression codes (from project 1) the matrix inversion
algorithm with our own SGD code. You can choose whether you want to
add the momentum SGD optionality or other SGD variants such as RMSprop
or ADAgrad or ADAM. The lecture notes from \href{{https://compphysics.github.io/MachineLearning/doc/pub/week40/html/week40.html}}{week 40 contain more
details}

Perform an analysis of the results for OLS and Ridge regression as
function of the chosen learning rates, the number of mini-batches and
epochs as well as algorithm for scaling the learning rate. You can
also compare your own results with those that can be obtained using
for example \textbf{Scikit-Learn}'s various SGD options.  Discuss your
results. For Ridge regression you need now to study the results as functions of  the hyper-parameter $\lambda$ and 
the learning rate $\gamma$.  Discuss your results.

You will need your SGD code for the setup of the Neural Network and Logistic Regression codes. You will find the Python \href{{https://seaborn.pydata.org/generated/seaborn.heatmap.html}}{Seaborn package} useful when plotting the results as function of the learning rate $\eta$ and the hyper-parameter $\lambda$ when you use Ridge regression.

\paragraph{Part b): Writing your own Neural Network code.}
Your aim now, and this is the central part of this project, is to
write your own Feed Forward Neural Network  code implementing the back
propagation algorithm discussed in the lecture slides from \href{{https://compphysics.github.io/MachineLearning/doc/pub/week41/html/week41.html}}{week 41}.

We will focus on a regression problem first and study either the
Franke function or terrain data (or both or other data sets) from
project 1.  Discuss again your choice of cost function.

Write an FFNN code for regression with a flexible number of hidden
layers and nodes using the Sigmoid function as activation function for
the hidden layers. Initialize the weights using a normal
distribution. How would you initialize the biases? And which
activation function would you select for the final output layer?

Train your network and compare the results with those from your OLS and Ridge Regression codes from project 1. 
You should test your results against a similar code using \textbf{Scikit-Learn} (see the examples in the above lecture notes from week 41) or \textbf{tensorflow/keras}. 

Comment your results and give a critical discussion of the results
obtained with the Linear  Regression code and your own Neural Network
code.  Compare the results with those from project 1.
Make an analysis of the regularization parameters and the learning rates employed to find the optimal MSE and $R2$ scores.

A useful reference on the back progagation algorithm is \href{{http://neuralnetworksanddeeplearning.com/}}{Nielsen's
book}. It is an excellent
read.



\paragraph{Part c): Testing different activation functions.}
You should now also test different activation functions for the hidden layers. Try out the Sigmoid, the RELU and the Leaky RELU functions and discuss your results. You may also study the way you initialize your weights and biases.

\paragraph{Part d): Classification  analysis using neural networks.}
With a well-written code it should now be easy to change the
activation function for the output layer.

Here we will change the cost function for our neural network code
developed in parts b) and c) in order to perform a classification analysis. 

We will here study the Wisconsin Breast Cancer  data set. This is a typical binary classification problem with just one single output, either True or Fale, $0$ or $1$ etc.
You find more information about this at the \href{{https://scikit-learn.org/stable/modules/generated/sklearn.datasets.load_breast_cancer.html}}{Scikit-Learn
site} or at the \href{{https://archive.ics.uci.edu/ml/datasets/breast+cancer+wisconsin+(original)}}{University of California
at Irvine}. 


To measure the performance of our classification problem we use the
so-called \emph{accuracy} score.  The accuracy is as you would expect just
the number of correctly guessed targets $t_i$ divided by the total
number of targets, that is 


\[ 
\text{Accuracy} = \frac{\sum_{i=1}^n I(t_i = y_i)}{n} ,
\]

where $I$ is the indicator function, $1$ if $t_i = y_i$ and $0$
otherwise if we have a binary classification problem. Here $t_i$
represents the target and $y_i$ the outputs of your FFNN code and $n$ is simply the number of targets $t_i$.

Discuss your results and give a critical analysis of the various parameters, including hyper-parameters like the learning rates and the regularization parameter $\lambda$ (as you did in Ridge Regression), various activation functions, number of hidden layers and nodes and activation functions.  


As stated in the introduction, it can also be useful to study other
datasets. 

Again, we strongly recommend that you compare your own neural Network
code for classification and pertinent results against a similar code using \textbf{Scikit-Learn}  or \textbf{tensorflow/keras} or \textbf{pytorch}.





\paragraph{Part e): Write your Logistic Regression code, final step.}
Finally, we want to compare the FFNN code we have developed with
Logistic regression, that is we wish to compare our neural network
classification results with the results we can obtain with another
method.

Define your cost function and the design matrix before you start writing your code.
Write thereafter a Logistic regression code using your SGD algorithm. You can also use standard gradient descent in this case, with a learning rate as hyper-parameter.
Study the results as functions of the chosen learning rates.
Add also an $l_2$ regularization parameter $\lambda$. Compare your results with those from your FFNN code as well as those obtained using \textbf{Scikit-Learn}'s logistic regression functionality.

The weblink  here \href{{https://medium.com/ai-in-plain-english/comparison-between-logistic-regression-and-neural-networks-in-classifying-digits-dc5e85cd93c3}}{\nolinkurl{https://medium.com/ai-in-plain-english/comparison-between-logistic-regression-and-neural-networks-in-classifying-digits-dc5e85cd93c3}}compares logistic regression and FFNN using the so-called MNIST data set. You may find several useful hints and ideas from this article. 


\paragraph{Part f) Critical evaluation of the various algorithms.}
After all these glorious calculations, you should now summarize the
various algorithms and come with a critical evaluation of their pros
and cons. Which algorithm works best for the regression case and which
is best for the classification case. These codes can also be part of
your final project 3, but now applied to other data sets.




\subsection*{Background literature}

\begin{enumerate}
\item The text of Michael Nielsen is highly recommended, see \href{{http://neuralnetworksanddeeplearning.com/}}{Nielsen's book}. It is an excellent read.

\item The textbook of \href{{https://www.springer.com/gp/book/9780387848570}}{Trevor Hastie, Robert Tibshirani, Jerome H. Friedman, The Elements of Statistical Learning, Springer}, chapters 3 and 7 are the most relevant ones for the analysis here. 

\item \href{{https://arxiv.org/abs/1803.08823}}{Mehta et al, arXiv 1803.08823}, \emph{A high-bias, low-variance introduction to Machine Learning for physicists}, ArXiv:1803.08823.
\end{enumerate}

\noindent
\subsection*{Introduction to numerical projects}

Here follows a brief recipe and recommendation on how to write a report for each
project.

\begin{itemize}
  \item Give a short description of the nature of the problem and the eventual  numerical methods you have used.

  \item Describe the algorithm you have used and/or developed. Here you may find it convenient to use pseudocoding. In many cases you can describe the algorithm in the program itself.

  \item Include the source code of your program. Comment your program properly.

  \item If possible, try to find analytic solutions, or known limits in order to test your program when developing the code.

  \item Include your results either in figure form or in a table. Remember to        label your results. All tables and figures should have relevant captions        and labels on the axes.

  \item Try to evaluate the reliabilty and numerical stability/precision of your results. If possible, include a qualitative and/or quantitative discussion of the numerical stability, eventual loss of precision etc.

  \item Try to give an interpretation of you results in your answers to  the problems.

  \item Critique: if possible include your comments and reflections about the  exercise, whether you felt you learnt something, ideas for improvements and  other thoughts you've made when solving the exercise. We wish to keep this course at the interactive level and your comments can help us improve it.

  \item Try to establish a practice where you log your work at the  computerlab. You may find such a logbook very handy at later stages in your work, especially when you don't properly remember  what a previous test version  of your program did. Here you could also record  the time spent on solving the exercise, various algorithms you may have tested or other topics which you feel worthy of mentioning.
\end{itemize}

\noindent
\subsection*{Format for electronic delivery of report and programs}

The preferred format for the report is a PDF file. You can also use DOC or postscript formats or as an ipython notebook file.  As programming language we prefer that you choose between C/C++, Fortran2008 or Python. The following prescription should be followed when preparing the report:

\begin{itemize}
  \item Use Canvas to hand in your projects, log in  at  \href{{https://www.uio.no/english/services/it/education/canvas/}}{\nolinkurl{https://www.uio.no/english/services/it/education/canvas/}} with your normal UiO username and password.

  \item Upload \textbf{only} the report file or the link to your GitHub/GitLab or similar typo of  repos!  For the source code file(s) you have developed please provide us with your link to your GitHub/GitLab or similar  domain.  The report file should include all of your discussions and a list of the codes you have developed.  Do not include library files which are available at the course homepage, unless you have made specific changes to them.

  \item In your GitHub/GitLab or similar repository, please include a folder which contains selected results. These can be in the form of output from your code for a selected set of runs and input parameters.
\end{itemize}

\noindent
Finally, 
we encourage you to collaborate. Optimal working groups consist of 
2-3 students. You can then hand in a common report. 



\subsection*{Software and needed installations}

If you have Python installed (we recommend Python3) and you feel pretty familiar with installing different packages, 
we recommend that you install the following Python packages via \textbf{pip} as
\begin{enumerate}
\item pip install numpy scipy matplotlib ipython scikit-learn tensorflow sympy pandas pillow
\end{enumerate}

\noindent
For Python3, replace \textbf{pip} with \textbf{pip3}.

See below for a discussion of \textbf{tensorflow} and \textbf{scikit-learn}. 

For OSX users we recommend also, after having installed Xcode, to install \textbf{brew}. Brew allows 
for a seamless installation of additional software via for example
\begin{enumerate}
\item brew install python3
\end{enumerate}

\noindent
For Linux users, with its variety of distributions like for example the widely popular Ubuntu distribution
you can use \textbf{pip} as well and simply install Python as 
\begin{enumerate}
\item sudo apt-get install python3  (or python for python2.7)
\end{enumerate}

\noindent
etc etc. 

If you don't want to install various Python packages with their dependencies separately, we recommend two widely used distrubutions which set up  all relevant dependencies for Python, namely
\begin{enumerate}
\item \href{{https://docs.anaconda.com/}}{Anaconda} Anaconda is an open source distribution of the Python and R programming languages for large-scale data processing, predictive analytics, and scientific computing, that aims to simplify package management and deployment. Package versions are managed by the package management system \textbf{conda}

\item \href{{https://www.enthought.com/product/canopy/}}{Enthought canopy}  is a Python distribution for scientific and analytic computing distribution and analysis environment, available for free and under a commercial license.
\end{enumerate}

\noindent
Popular software packages written in Python for ML are

\begin{itemize}
\item \href{{http://scikit-learn.org/stable/}}{Scikit-learn}, 

\item \href{{https://www.tensorflow.org/}}{Tensorflow},

\item \href{{http://pytorch.org/}}{PyTorch} and 

\item \href{{https://keras.io/}}{Keras}.
\end{itemize}

\noindent
These are all freely available at their respective GitHub sites. They 
encompass communities of developers in the thousands or more. And the number
of code developers and contributors keeps increasing.





% ------------------- end of main content ---------------

\end{document}

